\documentclass[]{article}
\usepackage{lmodern}
\usepackage{amssymb,amsmath}
\usepackage{ifxetex,ifluatex}
\usepackage{fixltx2e} % provides \textsubscript
\ifnum 0\ifxetex 1\fi\ifluatex 1\fi=0 % if pdftex
  \usepackage[T1]{fontenc}
  \usepackage[utf8]{inputenc}
\else % if luatex or xelatex
  \ifxetex
    \usepackage{mathspec}
  \else
    \usepackage{fontspec}
  \fi
  \defaultfontfeatures{Ligatures=TeX,Scale=MatchLowercase}
\fi
% use upquote if available, for straight quotes in verbatim environments
\IfFileExists{upquote.sty}{\usepackage{upquote}}{}
% use microtype if available
\IfFileExists{microtype.sty}{%
\usepackage{microtype}
\UseMicrotypeSet[protrusion]{basicmath} % disable protrusion for tt fonts
}{}
\usepackage[margin=1in]{geometry}
\usepackage{hyperref}
\hypersetup{unicode=true,
            pdftitle={Bayesian inference},
            pdfauthor={S Jin},
            pdfborder={0 0 0},
            breaklinks=true}
\urlstyle{same}  % don't use monospace font for urls
\usepackage{graphicx,grffile}
\makeatletter
\def\maxwidth{\ifdim\Gin@nat@width>\linewidth\linewidth\else\Gin@nat@width\fi}
\def\maxheight{\ifdim\Gin@nat@height>\textheight\textheight\else\Gin@nat@height\fi}
\makeatother
% Scale images if necessary, so that they will not overflow the page
% margins by default, and it is still possible to overwrite the defaults
% using explicit options in \includegraphics[width, height, ...]{}
\setkeys{Gin}{width=\maxwidth,height=\maxheight,keepaspectratio}
\IfFileExists{parskip.sty}{%
\usepackage{parskip}
}{% else
\setlength{\parindent}{0pt}
\setlength{\parskip}{6pt plus 2pt minus 1pt}
}
\setlength{\emergencystretch}{3em}  % prevent overfull lines
\providecommand{\tightlist}{%
  \setlength{\itemsep}{0pt}\setlength{\parskip}{0pt}}
\setcounter{secnumdepth}{0}
% Redefines (sub)paragraphs to behave more like sections
\ifx\paragraph\undefined\else
\let\oldparagraph\paragraph
\renewcommand{\paragraph}[1]{\oldparagraph{#1}\mbox{}}
\fi
\ifx\subparagraph\undefined\else
\let\oldsubparagraph\subparagraph
\renewcommand{\subparagraph}[1]{\oldsubparagraph{#1}\mbox{}}
\fi

%%% Use protect on footnotes to avoid problems with footnotes in titles
\let\rmarkdownfootnote\footnote%
\def\footnote{\protect\rmarkdownfootnote}

%%% Change title format to be more compact
\usepackage{titling}

% Create subtitle command for use in maketitle
\newcommand{\subtitle}[1]{
  \posttitle{
    \begin{center}\large#1\end{center}
    }
}

\setlength{\droptitle}{-2em}

  \title{Bayesian inference}
    \pretitle{\vspace{\droptitle}\centering\huge}
  \posttitle{\par}
    \author{S Jin}
    \preauthor{\centering\large\emph}
  \postauthor{\par}
      \predate{\centering\large\emph}
  \postdate{\par}
    \date{2019-02-11}


\begin{document}
\maketitle

\subsubsection{1 Two approaches}\label{two-approaches}

In frequentist inference, a parameter \(\theta\) is assumed as a fixed
unknown quantity, however, in Bayesian inference, we assume a parameter
\(\theta\) to be a random variable. For example, the coefficients in
linear regression model, or unknown population parameters, are random
variables.

\subsubsection{2 Bayes' rule}\label{bayes-rule}

Let

\begin{itemize}
\item
  \(\theta\) represents proportation of people who are mutants in
  Atlanta.
\item
  \(Y\) is the number of mutants from a random sample in Atlanta.
\end{itemize}

Before collecting and observing the data \(Y\), we have some beliefs (or
preknowledge) about \(\theta\), \(p(\theta)\); and some beliefs about
\(Y\) for given each value of \(\theta\), \(p(y|\theta)\). Then we
construct a joint density from

\begin{itemize}
\item
  \(p(\theta)\);
\item
  \(p(y|\theta)\).
\end{itemize}

After collecting and observing the data \(Y\), we update our
preknowledge (\(\theta\)) and have \(p(\theta|y)\), which is conditional
probability. According to the Bayes' rule,
\[p(\theta|y) = \frac{p(\theta,y)}{p(y)}=\frac{p(\theta)p(y|\theta)}{p(y)}=\frac{p(\theta)p(y|\theta)}{\int_{\theta}p(\theta)p(y|\theta)d\theta}.\]
As the \(p(y)\) does not rely on the random varialbe \(\theta|y\), we
can omit it and use proportion form, yielding \emph{unnormalized
posterior density} on the right side of following:
\[p(\theta|y)\propto p(\theta)p(y|\theta),\] where \(p(y|\theta)\) is
taken here as a function of \(\theta\), not of \(y\).


\end{document}
